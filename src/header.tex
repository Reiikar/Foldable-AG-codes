\usepackage{verbatim}
\usepackage[utf8]{inputenc}
\usepackage{pict2e}
\usepackage{tikz}

\usepackage{amsmath}
\usepackage{amsfonts}
\usepackage{amssymb}
\usepackage{amsthm}
\usepackage{mathtools} %pour mathclap
\usepackage{xcolor}
%\usepackage{authblk} %Multiples affiliations author\usepackage[all]{xy}
\usepackage{graphicx}
\usepackage{geometry}
\usepackage{hyperref} % hyperlinks
\geometry{hmargin=2.5cm,vmargin=1.5cm}
\usepackage{tikz-cd}


%----------- THÉORÈMES ET DÉFINITIONS -------------%
\theoremstyle{plain}
\newtheorem{theorem}{Theorem}[section]
\newtheorem*{theorem*}{Theorem}
\newtheorem{proposition}[theorem]{Proposition}
\newtheorem*{proposition*}{Proposition}
\newtheorem{corollary}[theorem]{Corollary}
\newtheorem*{corollary*}{Corollary}
\newtheorem{lemma}[theorem]{Lemma}
\newtheorem*{lemma*}{Lemma}
\newtheorem{claim}[theorem]{Claim}
\newtheorem{fact}[theorem]{Fact}
\newtheorem{hyp}{Hypothesis}

\newtheorem{definition}[theorem]{Definition} 
% Sarah: je change le style des définitions pour qu'elles soient en italiques car
% pour les longues définitions on ne sait pas où elles se terminent.
\theoremstyle{definition} 
\newtheorem{remark}[theorem]{Remark}
\newtheorem{example}[theorem]{Example}
\newtheorem*{example*}{Example}
\newtheorem{problem}[theorem]{Problem}
\newtheorem{construction}[theorem]{Construction}
\newtheorem{notation}[theorem]{Notation}
%--------------------------------------------------%

\DeclareMathOperator{\Span}{Span}
\DeclareMathOperator{\Pic}{Pic}
\DeclareMathOperator{\Supp}{Supp}
\DeclareMathOperator{\Aut}{Aut}
\DeclareMathOperator{\Div}{Div}

\newcommand\gen[1]{\left\langle #1 \right\rangle}
\newcommand\fold[1]{\textsf{\textbf{Fold}}\left[#1\right]}
\newcommand\poles[1]{\left( #1 \right)_\infty}
\newcommand\zeroes[1]{\left( #1 \right)_0}

\newcommand\mydef{\coloneqq}

\newcommand{\cd}{\cdot}
\newcommand{\C}{\mathbb{C}}
\newcommand{\N}{\mathbb{N}}
\newcommand{\Z}{\mathbb{Z}}
\newcommand{\Q}{\mathbb{Q}}
\newcommand{\R}{\mathbb{R}}

\newcommand{\ii}{i_{\rm{max}}}

\newcommand{\Floor}[1]{\left\lfloor#1\right\rfloor}
\newcommand{\F}{\mathbb{F}}
\newcommand{\calC}{\mathcal{C}}
\newcommand{\calF}{\mathcal{F}}
\newcommand{\calG}{\mathcal{G}}
\newcommand{\calH}{\mathcal{H}}
\newcommand{\calP}{\mathcal{P}}

\newcommand{\set}[1]{\left\{#1\right\}}
\newcommand{\size}[1]{\left|#1\right|}
\newcommand{\range}[1]{\set{0,\dots,#1-1}}
\newcommand{\Range}[2]{\set{#1,\dots,#2}}

%Commentaires :

\newcommand{\jade}[1]{{\color{blue!50!red}#1}}
\newcommand{\todo}[1]{{\color{red}TODO: #1}}
\newcommand{\fr}[1]{{\color{green!70!black}#1}}